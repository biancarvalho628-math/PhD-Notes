\documentclass[a4paper,english,11pt]{article}
\usepackage{babel}  %If I need to change the language
\usepackage{tikz-cd}    %For commutative diagrams
\usepackage{amsmath, amsthm, amssymb, enumerate, setspace}  %general purpose
\usepackage{fancyhdr} % Fancy header. ;-)
\usepackage{hyperref} % https://www.andy-roberts.net/writing/latex/pdfs

%Definition of ambients
\theoremstyle{definition}
\newtheorem{Def}{Definition}
\theoremstyle{plain}
\newtheorem{theorem}{Theorem}[section]
\newtheorem{prop}{Proposition}[section]
\newtheorem{corol}{Corollary}[section]
\newtheorem{lema}{Lemma}[section]
\theoremstyle{remark}
\newtheorem{ex}{\bf{Example}}
\newtheorem{rem}{\color{red}Remark}
\newtheorem*{nott}{Notation}
\newtheorem{axi}{Axiom}

%commands
    %categories
    \newcommand{\Top}{\textbf{\text{Top}}}
    \newcommand{\GT}{G\Top}
    \newcommand{\GTp}{\GT_{\ast}}
    \newcommand{\pSp}{\textbf{\text{pSp}}}
    \newcommand{\Sp}{\textbf{\text{Sp}}}
    \newcommand{\Ab}{\textbf{\text{Ab}}}
    \newcommand{\Hom}{\text{Hom}}
    %topological operators
    \newcommand{\Susp}{\Sigma^{\infty}}
    \newcommand{\susp}{\mathbf{\Sigma}}
    \newcommand{\Cyl}{\text{Cyl}}
    \newcommand{\Map}{\text{Map}}
    \newcommand{\lspec}{\mathbf{\Omega}}
    %important maps
    \newcommand{\struc}{\varepsilon}
    \newcommand{\astruc}{\bar{\varepsilon}}
\title{Notes on SHT}
\author{Bianca Carvalho de Oliveira}
\date{\today}
\begin{document}
\maketitle
\section{Spectra}
\begin{Def}[The Spectra Category]
    A \textit{spectrum} \(X\) is a family of based topoogical spaces \(\{X_n\:n\geq 0\}\) together with structure maps
    \[\struc_n:\Sigma X_n\to X_{n+1},\]
    in which \(\Sigma \) is the based suspension functor.

    A map between to spectra, \(f:X\to Y\), is a family of maps of based topological spaces, \(\{f_n:X_n\to Y_n:n\geq 0\}\), that agrees with the structure maps, that is, for each \(n\) we have a commutative square:
    \begin{equation*}
      \begin{tikzcd}
        \Sigma X_n \arrow{r}{\struc_n}\arrow{d}{f_n}& X_{n+1}\arrow{d}{f_{n+1}}\\
        \Sigma Y_n \arrow{r}{\struc_n}& Y_{n+1}
      \end{tikzcd}.
    \end{equation*}
    Having objects and maps we can define a category of prespectra, which we will denote as \(\pSp\). 
\end{Def}
Let \(X=(X_n:n\geq 0)\) be a spectrum. For each, \(n\) we have homotopy groups, \(\pi_k(X_n)\). Notice that the structure maps induce a map
\[\struc_{n\ast}:\pi_k(\Sigma X_n)\to\pi_k(X_{n+1}),\]
and by the definition of suspension, for each map
\[\alpha:S^k\to X_n,\]
we have a suspension
\[\Sigma\alpha:S^{k+1}\to \Sigma X_n.\]
Therefore, \(\Sigma\) induce maps
\[\Sigma_\ast:\pi_k(X_n)\to\pi_{k+1}(\Sigma X_n).\]
Adding the above information together, for each \(k\in\mathbb{Z}\), we get a sequence
\begin{equation*}
    \begin{tikzcd}
        \pi_k(X_0)\arrow{r}&\cdots\arrow{r}&\pi_{k+n}(X_n)\arrow{r}{\Sigma^\ast}&\pi_{k+n+1}(\Sigma X_n)\arrow{r}{\struc_{n\ast}}&\  \\
        \ &\ &\ \arrow{r}{\struc_{n\ast}}&\ \pi_{k+n+1}(X_{n+1})\arrow{r}&\cdots
    \end{tikzcd}
\end{equation*}
The colimit of this sequence will be called the \(k\)-th homotopy group of the spectrum \(X\), and denoted \(\pi_k(X)\).\\
The book asserts that for \(k<0\) the sequence is defined from \(n\geq|k|\). I would correct it as such: 
{\color{purple} The sequence is defined for \(n\geq \max\{2-k,0\}\), for in that case all groups are abelian groups}. 
There is no problem in not starting at \(0\) as this definition is meant to capture the asymptotical homotopical behavior. Would this be the same as taking the homotopy colimit of
\begin{equation}
  \label{hip1}
  \begin{tikzcd}
    \cdots\arrow{r}& X_n \arrow{r}{\Sigma}&\Sigma X_n\arrow{r}{\struc}&X_{n+1}\arrow{r}&\cdots
  \end{tikzcd}
\end{equation}
\begin{rem}
  For each \(k\), \(\pi_k\) is a functor from \(\pSp\) to \(\Ab\), the category of abelian groups.
\end{rem} 
If we were working with CW-complexes, this would be the moment we would discuss cells, but as we are not, we approximate this but a discussion of "elements".
\begin{Def}
  Given \(X=(X_n:n\geq 0)\) a spectrum, a \(k\) dimensional element of \(X\) is the set of maps
  \[\sigma:S^{k+n}X_n\to X_n,\]
  quotiented by the relation generated by
  \[\sigma_i\sim\sigma_j, \text{ if } \Sigma\sigma_i=\sigma_j.\]
  So now \(\pi_k(X)\) is the set of \(k\) dimensional elements of \(X\) up to homotopy.
\end{Def}  
\begin{rem}
  The suspension is given by smash product by \(S^1\) {\color{red} on the right}, i.e.
  \[\Sigma X=X\wedge S^1.\]
\end{rem}
Now we define a important type of spectrum.
\begin{Def}[Suspension Spectrum]
  Let \(X\) be a based topological space. De suspension spectrum over \(X\), \(\Susp X\), is defined levelwise as 
  \[\Susp X_n=\Sigma^{n}X=X\wedge S^n,\]
  with structure maps the canonical homeomorphisms
  \[X\wedge S^n\wedge S^1 \cong X\wedge S^{n+1}.\]
\end{Def}
We can "reverse" this process.
\begin{Def}[Shift desuspension spectrum]
  Given \(X\) a based topological space, and a natural number \(k\), the \textit{ \(k\) free spectrum}, or \textit{\(k\) shift desuspension spectrum} of \(X\), \(F_k X\), is given levelwise as
  \[F_kX_n=\begin{cases}
    \Sigma^{n-k}X,\text{ for }n\geq d,\\
    \ast,\text{ for } n<d,
  \end{cases}\]
  with the same structure maps of the suspension spectrum. 
\end{Def}  
Notice that
\[\pi_m(F_kX)=\pi_{m-k}(\Susp X).\]
NOw let's talk about homotopy theory.
\begin{Def}[homotopy of spectra]
  Let \(X=(X_n),\ Y=(Y_n)\) be two spectra, and \(f,g:\begin{tikzcd}[column sep =0.9em] X\arrow[shift left=1, shorten >=1.5]{r}\arrow[shift right=1, shorten >=1.5]{r}& Y \end{tikzcd}\) two maps of spectra. Let \(\Cyl(X)\) be the spectrum defined levelwise as 
  \[\Cyl(X)_n=X_n\wedge I_{+},\]
  in which the structure maps are
  \[\struc'_n=\struc_n\wedge 1_{I_{+}},\]
  \(\struc_n\) being the structure maps of \(X\).\\

  We say that \(f\) and \(g\) are homotopic, and write \(f\sim g\), if there is a map of spectra
  \[h:\Cyl(X)\to Y\]
  such that for each \(n\geq 0\),
  \[h_n:X_n\wedge I_{+}\to Y_n,\]
  is a homotopy of based topological spaces.
\end{Def}
\begin{Def}
  Let \(X=(X_n)\) and \(Y=(Y_n)\) be two spectra, and \(f:X\to Y\) be a map of spectra. We say that \(f\) is a \textit{stable equivalence}, or a \textit{\(\pi_\ast\)-isomorphism}, if, for each \(k\in\mathbb{Z}\), \(f_*:\pi_k(X)\to\pi_k(Y)\) is an isomorphism.\\

  Also, \(f\) is called a \textit{level equivalence} if, for each \(n\geq 0\), \(f_n:X_n\to Y_n\) is a weak homotopy equivalence.\\
  
  Lastly, \(f\) is called a \textit{homotopy equivalence} if there is a map \(g:Y\to X\) such that
  \[fg\sim 1_Y \text{ and } gf\sim 1_X,\]
  where \(\sim\) means that there is a homotopy of spectra.
\end{Def}
And stable homotopy theory will focus, as the names suggests, on stable equivalences. 
Notice that if \(f\) is a homotopy equivalence, then it is a level equivalence. And if \(f\) is a level equivalence, then it is a stable equivalence.\\

We now have the language to better express the idea that the stable homotopy groups are meant to capture the asymptotical homotopy behaviour of spectra.

\begin{Def}
  Let \(X=(X_n)\) be a spectrum, and \(k\geq 0\). We define the \(k\) eventual spectrum, \(X^k\) levelwise as 
  \[X^k_n=\begin{cases}
    X_n,\text{ if }n\geq k;\\
    \ast,\text{ if } 0\leq n < k.
  \end{cases}\]
\end{Def}
\begin{prop}
  Let \(X\) be a spectrum, and \(k\geq 0\). The inclusion \(X^k\subseteq X\) is a stable equivalence.
\end{prop}
\begin{proof}
  Let's begin by proving that \(\pi_m(X^k)=\pi_m(X)\) for all \(m\in\mathbb{Z}\). Indeed, for each \(n\geq 0\), we have maps
  \[\pi_{m+n}(X_n)\to\pi_m(X),\]
  that factor any map leaving \(\pi_{m+n}(X_n)\). And, as 
  \[\pi_{m+n'}(X^k_{n'})=\pi_{m+n'}(\ast)=\ast, n'<k\]
  any map leaving \(\pi_{m+n'}(X^k_n)\) factor trough 
  \[\pi_{m+n'}(X^k_{n'})\to\pi_m(X).\]
  That is, \(\pi_m(X)\) is the colimit of the sequence that defines \(\pi_m(X^k)\), and therefore they are isomorphic. 
  NOw let's look at the inclusion 
  \[\iota:X^k\to X.\]
  It is given levelwise as
  \[\iota_n=\begin{cases}
    0,\text{ if }n<k;\\
    1_{X_n},\text{ if } n\geq k.
  \end{cases}\]
  so that it induces the identity isomorphism, or the \(0\) morphism in each homotopy group. By the same reasoning about the \(0\) map, the identity on \(\pi_m(X)\) is the desired colimit, and therefore the inclusion is a stable equivalence.
\end{proof}
\section{Operations on Spectra}
If we want to work in the category of spectra define above, we should be able to define some operations in it. 
\begin{Def}[Coproduct of spectra]
  Let \(X=(X_n)\) and \(Y=(Y_n)\) be two spectra. The \textit{coproduct} of \(X\) and \(Y\), \(X\vee Y\), is defined levelwise as
  \[(X\vee Y)_n=X_n\vee Y_n,\]
  with structure maps the canonical homeomorphisms
  \[(X_n\vee Y_n)\wedge S^1\cong (X_n\wedge S^1)\vee (Y_n\wedge S^1)\xrightarrow{\struc_n\vee \struc'_n} X_{n+1}\vee Y_{n+1},\]
  where \(\struc_n\) and \(\struc'_n\) are the structure maps of \(X\) and \(Y\), respectively.
\end{Def}
Notice that the first map is an isomorphism because the suspension functor is a left adjoint, and as such commutes with colimts. Dually, we can define the product of spectra.
\begin{Def}[Product of spectra]
  Let \(X=(X_n)\) and \(Y=(Y_n)\) be two spectra. The \textit{product} of \(X\) and \(Y\), \(X\times Y\), is defined levelwise as
  \[(X\times Y)_n=X_n\times Y_n,\]
  with structure maps the canonical homeomorphisms
  \[(X_n\times Y_n)\wedge S^1\longrightarrow (X_n\wedge S^1)\times (Y_n\wedge S^1)\xrightarrow{\struc_n\times \struc'_n} X_{n+1}\times Y_{n+1},\]
  where \(\struc_n\) and \(\struc'_n\) are the structure maps of \(X\) and \(Y\), respectively.
  In this case, we hae that the adjoint structure map is 
  \[X_n\times Y_n\xrightarrow{\astruc_n\times\astruc'_n}\Omega X_{n+1}\times\Omega Y_{n+1}\xrightarrow{\cong}\Omega(X_{n+1}\times Y_{n+1}).\]
\end{Def}
\begin{rem}
  When working with colimits it's more natural to use the normal structure map, or suspension bonding map, while when working with limits it's more natural to use the adjoint structure map, or loop bonding map.
\end{rem}
Another useful pair of constructions are the \textit{smash product} and the \textit{function spectrum}.
\begin{Def}[Smash product]
  Let \(X=(X_n)\) be a spectrum, and \(K\) be a based topological space. The smash product of \(K\), and\(X\) is the spectra, \(K\wedge X\), defined levelwise as 
  \[(K\wedge X)_n=K\wedge X_n,\]
  and the structure maps are
  \[K\wedge X\xrightarrow{1_K\wedge\struc_n}K\wedge X_{n+1}.\]
  One example of such spectrum is the reduced suspension spectrum of \(X\), \(\susp X=S^1\wedge X\).  
\end{Def}
\begin{rem}
  Now the motivation to define the suspension functor by applying \(S^1\) on the right becomes clear. If we were to apply it on the left we would need to shuffle the two spheres before applying the structure maps. But the real condition is that the two spheres are on different sides, and the choice of which sphere is in each side is a question of {\color{red} preference}.
\end{rem}
\begin{Def}[Functions spectrum]
  Let \(X=(X_n)\) be a spectrum, and \(K\) be a based topological space. The function spectrum from \(K\) to \(X\), \(F(K,X)\), is defined levelwise as
  \[F(K,X)_n= \Map_\ast(K,X_n),\]
  the space of based maps from \(K\) to \(X_n\). The structure maps are given by the adjunction
  \[\Map_\ast(K,X_n)\xrightarrow{\Map_\ast(K,\astruc_n)}\Map_\ast(K,\Omega X_{n+1})\xrightarrow{\cong}\Omega \Map_\ast(K,X_{n+1}).\]
  One example of such spectrum is the loop spectrum of \(X\), \(\lspec X=F(S^1,X)\).
\end{Def}
\begin{rem}
  In this case we wil need to use a shuffling map to define the adjoint structure map, as the spheres appear in reverse order in 
  \[\Map_\ast(S^1,\Map_\ast(S^1,X_{n+1})).\] 
\end{rem}
\begin{prop}
  The operations \(K\wedge\_\), and \(F(K,\_)\) are adjoint.  
\end{prop}
\begin{proof}
  Indeed, we have an adjunction at each level. We have to show that at the collection of maps in the natural isomorphism is a map of spectra, i.e. If
  \[\Phi_n:\Top_\ast(K\wedge X_n,Y_n)\to \Top_\ast(X_n,F(K,Y_n)),\]
  is the isomorphism, so, given \(f:K\wedge X_n\to Y_n\), a map of spectra, \(\Phi_n(f):X_n\to F(K,Y_n)\) is a map of spectra. That is, we need to show that 
\begin{equation*}
    \begin{tikzcd}
      X_n\arrow{rr}{\Phi_n(f)}\arrow{d}{\astruc_n}&\ & F(K,Y_n)\arrow{d}{\astruc_{F,n}}\\
      \Omega X_{n+1}\arrow{rr}{\Omega\Phi_{n+1}(f)}&\ & F(K,\Omega Y_{n+1})
    \end{tikzcd}
\end{equation*}
commutes. But indeed, in the leftmost path we have the map
\[\astruc_F\bar{f}_{n}\]
%\begin{align*}
 % X_n &\to F(K,\Omega Y_{n+1})\cong \Omega F(K,Y_{n+1}) \\
  %x &\mapsto \astruc_F(\varphi_{f,x}):K\to \Omega Y_{n+1}\\
  %&\ \ \ \ \ \ \ \ \ \ \ \ \ \ \ \ \ \ k \mapsto \astruc_{Y} (f(k,x)),
%\end{align*}
and the rightmost path is
\[\Omega\bar{f}_{n+1}\astruc_X.\]
What we know to be true is that 
\[f_{n+1}\struc_K=\struc_Y\Sigma f_{n}\]. Let us look at the adjunctions we have.Fisrt,, we know that 
\[\Hom(K\wedge X_n, Y_n)\cong \Hom(X_n,\Map(K,Y_n)): f_n\leftrightarrow\bar{f}_n.\]
By naturality on the second variable,and using the map\(\astruc_Y:Y_n\to\Omega Y_{n+1}\), we have
\[\Hom(K\wedge X_n,\Omega Y_{n+1})\cong\Hom(X_n,\Map(K,\Omega Y_{n+1})):\astruc_Yf_n\leftrightarrow\astruc_F\bar{f}_n,\]
beacause \(\astruc_F=\Map(K,\astruc_Y)\). Now, using the \(\Omega-\Sigma\)adjunction:
\[\Hom(Y_n,\Omega Y_{n+1})\cong \Hom(\Sigma Y_n,Y_{n+1}):\astruc_Y\leftrightarrow \struc_Y,\]
using the naturality now in the firstvariable,  with the map \(f_n:K\wedge X_n\to Y_n\),we obtain
\[\Hom(K\wedge X_n,\Omega\cong \Hom(\Sigma) K\wedge X_n,Y_{n+1}): \astruc_Yf_n\leftrightarrow \struc_Y\Sigma f_n.\]
This means that the adjoint of \(\astruc_F\bar{f}_n\)is \(\struc_Y\Sigma f_n\).But, we know that \(\struc_Y\Sigma f_n=f_{n+1}\struc_K\). lets see what is its adjoint. First, we have
\[\Hom(K\wedge X_{n+1},Y_{n+1})\cong\Hom(X_{n+1},\Map(K,Y_{n+1})):f_{n+1}\leftrightarrow\bar{f}_{n+1},\]
and by naturality in the first variable, using the map \(\struc_X:\Sigma X_n\to X_{n+1}\), we have
\[\Hom(\Sigma X_n,\Map(K,Y_{n+1}))\cong\Hom(\Sigma K\wedge X_n,Y_{n+1}): \bar{f}_{n+1}\struc_X\leftrightarrow f_{n+1}\struc_K,\]
since \(\struc_K=1_k\wedge\struc_X\). Now, using the \(\Omega-\Sigma\) adjunction again, we have
\[\Hom(\Sigma X_n,X_{n+1})\cong\Hom(X_n,\Omega X_{n+1}):\struc_X\leftrightarrow\astruc_X,\]
and by naturality in the second variable, using the map \(\bar{f}_{n+1}:X_{n+1}\to \Map(K,Y_{n+1})\), we have 
\[\Hom(\Sigma X_n,\Map(K,Y_{n+1}))\cong\Hom(X_n,\Map(K,\Omega Y_{n+1})): \bar{f}_{n+1}\astruc_X\leftrightarrow\Omega\bar{f}_{n+1}\astruc_X.\]
Which means that the adjoint of \(f_{n+1}\struc_K\) is \(\Omega\bar{f}_{n+1}\astruc_X\). \\
%\begin{align*}
%  X_n &\to \Omega F(K,Y_n)\cong \Omega F(K,Y_{n+1}) \\
%  x &\mapsto \varphi_{f,\astruc(x)}:K\to \Omega Y_{n+1}\\
%  &\ \ \ \ \ \ \ \ \ \ \ \ \ \ \ \ \ k\mapsto f(k,\astruc(x)).
%\end{align*}
Therefore the original diagram commutes, and the operations are adjoint.

\end{proof}
\section{The smash product of spectra}
In this section we discuss the smash product of spectra. It is important as it gives the adequate category of spectra a symetric monoidal structure, allowing us to define ring spectra, module spectra and (co)homology  over spectra. But to begin we will describe what we want the smash product to be, without defining it, and deducing some properties from this.
\begin{Def}[The Smash product]
  Let us describe what we want the smash product of spectra to be. 
  \begin{enumerate}
    \item We have a category of spectra, \(\Sp\), not necessarily the same as \(\pSp\). In this category we have all the operaations we defined for the category \(\pSp\). Lastly, we have n equivalence of categories
    \[\text{Ho}\pSp\simeq\text{Ho}\Sp;\]
    \item \(\Sp\) is a simmetric monoidal category. The tensor operation is called \textit{smash product}, and denoted \(\wedge\);
    \item The unit in \(\Sp\) is the sphere spectrum, \(\mathbb{S}=\Susp S^0\);
    \item We have an isomorphism
    \[F_dA\wedge F_eB\cong F_{d+e}(A\wedge B);\]
    \item The smash product preserves cellular spectra, and all the stable equivalences between them;
  \end{enumerate}
\end{Def} 
\begin{rem}
  The is a number of remarks to be made about this definition.
  \begin{enumerate}
    \item smashing a spectrum with a suspension spectrum is given by the smash with the base space, i.e.:
    \[\Susp K\wedge X\cong K\wedge X,\]
    and we also have That
    \[\Susp X\wedge \Susp Y\cong \Susp(X\wedge Y);\]
    \item By item \(4\) we have that the smash product of cellular spectra is a cellular spectrum;
    \item By item \(6\) it is posible to show that we can define a smash product in \(\text{Ho}\Sp\), with
    \[X\wedge^{\mathbb{L}}Y=QX\wedge QY,\]
    on which \(QA\) is the cellular replacement of \(A\). This is the derived smash product;
  \end{enumerate}
\end{rem}
\begin{Def}[Ring Spectrum]
  A \textit{ring spectrum} is a monoid object in \((\Sp,\wedge, \mathbb{S})\), i.e. a spectrum \(R\) together with maps
  \[\mu:R\wedge R\to R,\]
  and
  \[\eta:\mathbb{S}\to R,\]
  such that the following diagrams commute:
  \begin{equation*}
    \begin{tikzcd}
      R\wedge R\wedge R\arrow{r}{\mu\wedge 1_R}\arrow{d}{1_R\wedge \mu}& R\wedge R\arrow{d}{\mu}\\
      R\wedge R\arrow{r}{\mu}& R
    \end{tikzcd},
  \end{equation*}
  and
  \begin{equation*}
    \begin{tikzcd}
      R\cong \mathbb{S}\wedge R\arrow{r}{\eta\wedge 1_R}\arrow{dr}[swap]{1_R}& R\wedge R\arrow{d}{\mu}&R\cong R\wedge \mathbb{S}\arrow{l}[swap]{1_R\wedge \eta}\arrow{dl}{1_R}\\
      &R&
    \end{tikzcd}.
  \end{equation*}

\end{Def}
\end{document}