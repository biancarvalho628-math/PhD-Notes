\documentclass[a4paper,english,11pt]{article}
\usepackage{babel}  %If I need to change the language
\usepackage{tikz-cd}    %For commutative diagrams
\usepackage{amsmath, amsthm, amssymb, enumerate, setspace}  %general purpose
\usepackage{fancyhdr} % Fancy header. ;-)
\usepackage{hyperref} % https://www.andy-roberts.net/writing/latex/pdfs

%Definition of ambients
\theoremstyle{definition}
\newtheorem{Def}{Definition}
\theoremstyle{plain}
\newtheorem{theorem}{Theorem}[section]
\newtheorem{prop}{Proposition}[section]
\newtheorem{corol}{Corollary}[section]
\newtheorem{lema}{Lemma}[section]
\theoremstyle{remark}
\newtheorem{ex}{\bf{Example}}
\newtheorem{rem}{\color{red}Remark}
\newtheorem*{nott}{Notation}
\newtheorem{axi}{Axiom}

%commands
    %categories
    \newcommand{\Top}{\textbf{\text{Top}}}
    \newcommand{\GT}{G\Top}
    \newcommand{\GTp}{\GT_{\ast}}
    \newcommand{\pSp}{\textbf{\text{pSp}}}
    %topological operators
    \newcommand{\Susp}{\Sigma^{\infty}}
\title{Notes on SHT}
\author{Bianca Carvalho de Oliveira}
\date{\today}
\begin{document}
\maketitle
\section{Spectra}
\begin{Def}[The Spectra Category]
    A \textit{spectrum} \(X\) is a family of based topoogical spaces \(\{X_n\:n\geq 0\}\) together with structure maps
    \[\varepsilon_n:\Sigma X_n\to X_{n+1},\]
    in which \(\Sigma X\) is the based suspension functor.

    A map between to spectra, \(f:X\to Y\), is a family of maps of based topological spaces, \(\{f_n:X_n\to Y_n:n\geq 0\}\), that agrees with the structure maps, that is, for each \(n\) we have a commutative square:
    \begin{equation*}
      \begin{tikzcd}
        \Sigma X_n \arrow{r}{\varepsilon}\arrow{d}{f_n}& X_{n+1}\arrow{d}{f_{n+1}}\\
        \Sigma Y_n \arrow{r}{\varepsilon}& Y_{n+1}
      \end{tikzcd}.
    \end{equation*}
    Having objects and maps we can define a category of prespectra, which we will denote as \(\pSp\). 
\end{Def}
Let \(X=(X_n:n\geq 0)\) be a spectrum. For each, \(n\) we have homotopy groups, \(\pi_k(X_n)\). Notice that the structure maps induce a map
\[\varepsilon_n^{\ast}:\pi_k(\Sigma X_n)\to\pi_k(X_{n+1}),\]
and by the definition of suspension, for each map
\[\alpha:S^k\to X_n,\]
we have a suspension
\[\Sigma\alpha:S^{k+1}\to \Sigma X_n.\]
Therefore, \(\Sigma\) induce maps
\[\Sigma^\ast:\pi_k(X_n)\to\pi_{k+1}(\Sigma X_n).\]
Adding the above information together, for each \(k\in\mathbb{Z}\), we get a sequence
\begin{equation*}
    \begin{tikzcd}
        \pi_k(X_0)\arrow{r}&\cdots\arrow{r}&\pi_{k+n}(X_n)\arrow{r}{\Sigma^\ast}&\pi_{k+n+1}(\Sigma X_n)\arrow{r}{\varepsilon_n^\ast}&\pi_{k+n+1}(X_{n+1})\arrow{r}&\cdots
    \end{tikzcd}
\end{equation*}
The colimit of this sequence will be called the \(k\)-th homotopy group of the spectrum \(X\), and denoted \(\pi_k(X)\).\\
The book asserts that for \(k<0\) the sequence is defined from \(n\geq|k|\). I would correct it as such: 
{\color{purple} The sequence is defined for \(n\geq 2-k\), for in that case all groups are abelian groups}. 
There is no problem in not starting at \(0\) as this definition is meant to capture the asymptotical homotopical behavior. Would this be the same as taking the homotopy colimit of
\begin{equation*}
  \begin{tikzcd}
    \cdots\arrow{r}& X_n \arrow{r}{\Sigma}&\Sigma X_n\arrow{r}{\varepsilon}&X_{n+1}\arrow{r}&\cdots
  \end{tikzcd}
\end{equation*}
If we were working with CW-complexes, this would be the moment we would discuss cells, but as we are not, we approximate this but a discussion of "elements".
\begin{Def}
  Given \(X=(X_n:n\geq 0)\) a spectrum, a \(k\) dimensional element of \(X\) is the set of maps
  \[\sigma:S^{k+n}X_n\to X_n,\]
  quotiented by the relation generated by
  \[\sigma_i\sim\sigma_j, \text{ if } \Sigma\sigma_i=\sigma_j.\]
  So now \(\pi_k(X)\) is the set of \(k\) dimensional elements of \(X\) up to homotopy.
\end{Def}  
\begin{rem}
  The suspension is given by smash product by \(S^1\) {\color{red} on the right}, i.e.
  \[\Sigma X=X\wedge S^1.\]
\end{rem}
  Now we define a important type of spectrum.
  \begin{Def}[Suspension Spectra]
    Let \(X\) be a based topological space. De suspension spectra over \(X\), \(\Susp X\), is defined levelwise as 
    \[\Susp X_n=\Sigma^{n}X=X\wedge S^n,\]
    with structure maps the canonical homeomorphisms
    \[X\wedge S^n\wedge S^1 \cong X\wedge S^{n+1}.\]
  \end{Def}
\end{document}