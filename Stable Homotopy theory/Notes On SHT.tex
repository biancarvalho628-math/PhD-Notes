\documentclass[a4paper,english,11pt]{article}
\usepackage{babel}  %If I need to change the language
\usepackage{tikz-cd}    %For commutative diagrams
\usepackage{amsmath, amsthm, amssymb, enumerate, setspace}  %general purpose
\usepackage{fancyhdr} % Fancy header. ;-)
\usepackage{hyperref} % https://www.andy-roberts.net/writing/latex/pdfs

%Definition of ambients
\theoremstyle{definition}
\newtheorem{Def}{Definition}
\theoremstyle{plain}
\newtheorem{theorem}{Theorem}[section]
\newtheorem{prop}{Proposition}[section]
\newtheorem{corol}{Corollary}[section]
\newtheorem{lema}{Lemma}[section]
\theoremstyle{remark}
\newtheorem{ex}{\bf{Example}}
\newtheorem{rem}{\color{red}Remark}
\newtheorem*{nott}{Notation}
\newtheorem{axi}{Axiom}

%commands
    %categories
    \newcommand{\Top}{\textbf{\text{Top}}}
    \newcommand{\GT}{G\Top}
    \newcommand{\GTp}{\GT_{\ast}}
    \newcommand{\pSp}{\textbf{\text{pSp}}}
    \newcommand{\Ab}{\textbf{\text{Ab}}}
    %topological operators
    \newcommand{\Susp}{\Sigma^{\infty}}
    \newcommand{\Cyl}{\text{Cyl}}
\title{Notes on SHT}
\author{Bianca Carvalho de Oliveira}
\date{\today}
\begin{document}
\maketitle
\section{Spectra}
\begin{Def}[The Spectra Category]
    A \textit{spectrum} \(X\) is a family of based topoogical spaces \(\{X_n\:n\geq 0\}\) together with structure maps
    \[\varepsilon_n:\Sigma X_n\to X_{n+1},\]
    in which \(\Sigma \) is the based suspension functor.

    A map between to spectra, \(f:X\to Y\), is a family of maps of based topological spaces, \(\{f_n:X_n\to Y_n:n\geq 0\}\), that agrees with the structure maps, that is, for each \(n\) we have a commutative square:
    \begin{equation*}
      \begin{tikzcd}
        \Sigma X_n \arrow{r}{\varepsilon_n}\arrow{d}{f_n}& X_{n+1}\arrow{d}{f_{n+1}}\\
        \Sigma Y_n \arrow{r}{\varepsilon_n}& Y_{n+1}
      \end{tikzcd}.
    \end{equation*}
    Having objects and maps we can define a category of prespectra, which we will denote as \(\pSp\). 
\end{Def}
Let \(X=(X_n:n\geq 0)\) be a spectrum. For each, \(n\) we have homotopy groups, \(\pi_k(X_n)\). Notice that the structure maps induce a map
\[\varepsilon_{n\ast}:\pi_k(\Sigma X_n)\to\pi_k(X_{n+1}),\]
and by the definition of suspension, for each map
\[\alpha:S^k\to X_n,\]
we have a suspension
\[\Sigma\alpha:S^{k+1}\to \Sigma X_n.\]
Therefore, \(\Sigma\) induce maps
\[\Sigma_\ast:\pi_k(X_n)\to\pi_{k+1}(\Sigma X_n).\]
Adding the above information together, for each \(k\in\mathbb{Z}\), we get a sequence
\begin{equation*}
    \begin{tikzcd}
        \pi_k(X_0)\arrow{r}&\cdots\arrow{r}&\pi_{k+n}(X_n)\arrow{r}{\Sigma^\ast}&\pi_{k+n+1}(\Sigma X_n)\arrow{r}{\varepsilon_{n\ast}}&\  \\
        \ &\ &\ \arrow{r}{\varepsilon_{n\ast}}&\ \pi_{k+n+1}(X_{n+1})\arrow{r}&\cdots
    \end{tikzcd}
\end{equation*}
The colimit of this sequence will be called the \(k\)-th homotopy group of the spectrum \(X\), and denoted \(\pi_k(X)\).\\
The book asserts that for \(k<0\) the sequence is defined from \(n\geq|k|\). I would correct it as such: 
{\color{purple} The sequence is defined for \(n\geq \max\{2-k,0\}\), for in that case all groups are abelian groups}. 
There is no problem in not starting at \(0\) as this definition is meant to capture the asymptotical homotopical behavior. Would this be the same as taking the homotopy colimit of
\begin{equation}
  \label{hip1}
  \begin{tikzcd}
    \cdots\arrow{r}& X_n \arrow{r}{\Sigma}&\Sigma X_n\arrow{r}{\varepsilon}&X_{n+1}\arrow{r}&\cdots
  \end{tikzcd}
\end{equation}
\begin{rem}
  For each \(k\), \(\pi_k\) is a functor from \(\pSp\) to \(\Ab\), the category of abelian groups.
\end{rem} 
If we were working with CW-complexes, this would be the moment we would discuss cells, but as we are not, we approximate this but a discussion of "elements".
\begin{Def}
  Given \(X=(X_n:n\geq 0)\) a spectrum, a \(k\) dimensional element of \(X\) is the set of maps
  \[\sigma:S^{k+n}X_n\to X_n,\]
  quotiented by the relation generated by
  \[\sigma_i\sim\sigma_j, \text{ if } \Sigma\sigma_i=\sigma_j.\]
  So now \(\pi_k(X)\) is the set of \(k\) dimensional elements of \(X\) up to homotopy.
\end{Def}  
\begin{rem}
  The suspension is given by smash product by \(S^1\) {\color{red} on the right}, i.e.
  \[\Sigma X=X\wedge S^1.\]
\end{rem}
Now we define a important type of spectrum.
\begin{Def}[Suspension Spectrum]
  Let \(X\) be a based topological space. De suspension spectrum over \(X\), \(\Susp X\), is defined levelwise as 
  \[\Susp X_n=\Sigma^{n}X=X\wedge S^n,\]
  with structure maps the canonical homeomorphisms
  \[X\wedge S^n\wedge S^1 \cong X\wedge S^{n+1}.\]
\end{Def}
We can "reverse" this process.
\begin{Def}[Shift desuspension spectrum]
  Given \(X\) a based topological space, and a natural number \(k\), the \textit{ \(k\) free spectrum}, or \textit{\(k\) shift desuspension spectrum} of \(X\), \(F_k X\), is given levelwise as
  \[F_kX_n=\begin{cases}
    \Sigma^{n-k}X,\text{ for }n\geq d,\\
    \ast,\text{ for } n<d,
  \end{cases}\]
  with the same structure maps of the suspension spectrum. 
\end{Def}  
Notice that
\[\pi_m(F_kX)=\pi_{m-k}(\Susp X).\]
NOw let's talk about homotopy theory.
\begin{Def}[homotopy of spectra]
  Let \(X=(X_n),\ Y=(Y_n)\) be two spectra, and \(f,g:\begin{tikzcd}[column sep =0.9em] X\arrow[shift left=1, shorten >=1.5]{r}\arrow[shift right=1, shorten >=1.5]{r}& Y \end{tikzcd}\) two maps of spectra. Let \(\Cyl(X)\) be the spectrum defined levelwise as 
  \[\Cyl(X)_n=X_n\wedge I_{+},\]
  in which the structure maps are
  \[\varepsilon'_n=\varepsilon_n\wedge 1_{I_{+}},\]
  \(\varepsilon_n\) being the structure maps of \(X\).\\

  We say that \(f\) and \(g\) are homotopic, and write \(f\sim g\), if there is a map of spectra
  \[h:\Cyl(X)\to Y\]
  such that for each \(n\geq 0\),
  \[h_n:X_n\wedge I_{+}\to Y_n,\]
  is a homotopy of based topological spaces.
\end{Def}
\begin{Def}
  Let \(X=(X_n)\) and \(Y=(Y_n)\) be two spectra, and \(f:X\to Y\) be a map of spectra. We say that \(f\) is a \textit{stable equivalence}, or a \textit{\(\pi_\ast\)-isomorphism}, if, for each \(k\in\mathbb{Z}\), \(f_*:\pi_k(X)\to\pi_k(Y)\) is an isomorphism.\\

  Also, \(f\) is called a \textit{level equivalence} if, for each \(n\geq 0\), \(f_n:X_n\to Y_n\) is a weak homotopy equivalence.\\
  
  Lastly, \(f\) is called a \textit{homotopy equivalence} if there is a map \(g:Y\to X\) such that
  \[fg\sim 1_Y \text{ and } gf\sim 1_X,\]
  where \(\sim\) means that there is a homotopy of spectra.
\end{Def}
\end{document}